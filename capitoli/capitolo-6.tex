% !TEX encoding = UTF-8
% !TEX TS-program = pdflatex
% !TEX root = ../tesi.tex

%**************************************************************
\chapter{Conclusioni}
\label{cap:conclusioni}
%**************************************************************

\section{Analisi del prodotto ottenuto}
Il prodotto ottenuto, pronto per il rilascio in ambiente di produzione, come spiegato nell'analisi
comparativa non verrà usato come strumento di produzione.
\\\\
Questo prodotto verrà comunque conservato dal proponente nel proprio repository GitHub, come strumento
da cui prendere spunto per un eventuale progetto futuro in NestJS.
\\\\
Il progetto realizzato ha seguito tutte le buone pratiche di sviluppo suggerite nella documentazione ufficiale
NestJS e vari design pattern.
\\\\
Come richiesto dalla proponente è stata implementata una suite di unit test automatici per testare tutti i servizi 
a livello branch.
\\\\
Il prodotto realizzato è molto facile da avviare, infatti qualsiasi sviluppatore che non conosce NestJS può andare 
nel repository GitHub del progetto e leggere il file README, che spiega in modo dettagliato come scaricare le 
dipendenze richieste tramite npm e far partite il servizio.

\section{Raggiungimento degli obiettivi}
Vediamo gli obiettivi pianificati e il loro stato di raggiungimento al termine dello stage: 
\\
Notazione:
\\
\begin{itemize}
    \item O per gli obiettivi obbligatori, vincolanti in quanto obiettivo primario richiesto dal proponente.
    \item D per gli obiettivi desiderabili, non vincolanti o strettamente necessari ma dal riconoscibile valore aggiunto.
    \item F per i vincoli facoltativi, rappresentante valore aggiunto non strettamente competitivo.
\end{itemize}
\leavevmode\newline
\begin{table}
    \begin{tabular}{|p{1.5cm}|p{7.7cm}|p{2cm}|} 
    \hline
    Codice & Descrizione  & Stato \\ 
    \hline
    O01 & Acquisizione competenze Java & Soddisfatto \\
    \hline
    O02 & Acquisizione competenze passaggio da monolite a microservizi & Soddisfatto \\
    \hline
    O03 & Acquisizione competenze framework Spring Boot & Soddisfatto \\
    \hline
    O04 & Acquisizione competenze strumento Spring Data JPA & Soddisfatto \\
    \hline
    O05 & Acquisizione competenze realizzazione servizio di REST API & Soddisfatto \\
    \hline
    O06 & Acquisizione competenze strumento Node.js & Soddisfatto \\
    \hline
    O07 & Acquisizione competenze framework Express.js & Soddisfatto \\
    \hline
    O08 & Realizzazione di un servizio REST prototipale con Spring Boot & Soddisfatto \\
    \hline
    O09 & Realizzazione di un servizio REST prototipale con Express.js & Soddisfatto \\
    \hline
    O10 & Effettuare un'analisi per il progetto Smart Parking & Soddisfatto \\
    \hline
    O11 & Realizzare una progettazione per il progetto Smart Parking & Soddisfatto \\
    \hline
    O12 & Realizzare una suite di test automatici con copertura a livello branch >= 90\% & Soddisfatto \\
    \hline
    O13 & Realizzare una suite di test automatici con copertura a livello linee di codice >= 60\% & Soddisfatto \\
    \hline
    O14 & Realizzare un numero di REST API progettate maggiore dell'80\% & Soddisfatto \\
    \hline
    O15 & Effettuare un'analisi comparativa tra la soluzione in Spring e in NestJS & Soddisfatto \\
    \hline
    O16 & Raggiungere gli obiettivi richiesti in autonomia seguendo il cronoprogramma & Soddisfatto \\
    \hline
    \end{tabular}
\end{table}

\leavevmode\newline
\begin{table}
    \begin{tabular}{|p{1.5cm}|p{7.7cm}|p{2cm}|} 
    \hline
    Codice & Descrizione  & Stato \\ 
    \hline
    D01 & Realizzare un numero di REST API progettate pari al 100\%  & Soddisfatto \\
    \hline
    \end{tabular}
\end{table}

\leavevmode\newline
\begin{table}
    \begin{tabular}{|p{1.5cm}|p{7.7cm}|p{2cm}|} 
    \hline
    Codice & Descrizione  & Stato \\ 
    \hline
    F01 & Analizzare come poter implementare le best practice dell'architettura a microservizi con NestJS & Soddisfatto parzialmente \\
    \hline
    \end{tabular}
\end{table}

\section{Valutazione personale}
L'esperienza di stage è stata molto positiva. La prima parte dello stage è stata quella più difficile da affrontare, 
in quanto ho dovuto imparare da zero delle tecnologie che non avevo mai visto come Spring e NestJS.
\\\\
Questo processo di apprendimento è stato per me molto utile, sia perché la conoscenza di questi due framework mi sarà molto
utile in ambito lavorativo, sia perché mi ha insegnato come apprendere il funzionamento di un framework in maniera abbastanza
dettagliata ed essere operativo in tempo breve.
\\\\
Ho potuto affacciarmi in un'azienda con molti dipendenti e quindi confrontarmi giornalmente con sviluppatori più esperti di me che mi hanno
aiutato molto.
\\\\
Sono soddisfatto del prodotto realizzato e di come si sono svolti i processi di sviluppo che hanno portato alla sua realizzazione. 
\\\\
Ritengo molto valida questo tipo di esperienza in azienda, anche per tutti i miei colleghi universitari futuri, in quanto mi ha 
fatto crescere molto come sviluppatore. 
\\
Anche se due mesi sono pochi ho potuto mettere mano a un progetto reale e sono molto cresciuto anche nella fase di analisi e progettazione.
\\\\
Sicuramente il background fornito dall'università è stato fondamentale per poter svolgere questo stage; in quanto l'università mi ha 
fornito tutti i concetti basilari di cui avevo bisogno per poter apprendere le nuove tecnologie e per poterle utilizzare in maniera 
corretta.
\\\\
L'università mi ha dato gli strumenti per poter analizzare con occhio critico le scelte progettuali che sono andato a svolgere.
\\\\
In conclusione sono pienamente soddisfatto dello stage che ho fatto e del prodotto che ho realizzato che, anche se non verrà 
utilizzato in produzione, ha posto comunque le basi per un futuro progetto realizzabile in NestJS e ha reso appetibile questa nuova tecnologia
per i progetti futuri del proponente.