% !TEX encoding = UTF-8
% !TEX TS-program = pdflatex
% !TEX root = ../Volpe_Andrea.tex

%**************************************************************
\chapter{Analisi comparativa}
\label{cap:analisi-comparativa}
%**************************************************************

\intro{In questo capitolo viene descritta la fase di analisi comparativa effettuata al termine
della realizzazione del progetto.}
\\\\\\\\\\\\\\
Il proponente ha deciso di realizzare questo progetto di stage con lo scopo di effettuare 
un'analisi comparativa tra la versione del prodotto finale in NestJS e quella in Spring.
\\\\
Essendo NestJS un framework abbastanza giovane (rilasciato nel 2016), il proponente si 
è interessato a questa tecnologia nell'ottica di poter sviluppatore futuri progetti 
aziendali con questo framework e sfruttarne i benefici.
\\\\
Avendo una bassa conoscenza della tecnologia è stato ritenuto troppo rischioso utilizzare NestJS per 
realizzare un nuovo progetto da zero. 
\\
E' stato deciso quindi di migrare un progetto già esistente, nell'ottica che se la soluzione
realizzata risultasse fallimentare, è comunque presente un progetto solido che può essere
messo in esercizio in ambiente di produzione, riducendo i costi del dover scartare un prodotto 
inutilizzabile.
\\\\
Il problema di dover scartare il prodotto realizzato in NestJS può presentarsi nel caso in cui in fase di sviluppo
sorgessero problemi di non fattibilità del software, non preventivati in fase di analisi per la
scarsa conoscenza del framework. Oppure per problemi legati alla soluzione finale realizzata (programma troppo pesante, prestazioni troppo lente ecc..), per lo 
stesso motivo di prima.

\section{Le due soluzioni}
Le due soluzioni sviluppano lo stesso prodotto con due framework diversi.
Il prodotto realizza un set di \gls{api} \gls{rest}, per la gestione delle operazioni \gls{crud} del
progetto Smart Parking.
\\\\
Il prodotto già esistente è stato realizzato col framework Spring, il nuovo col framework NestJS:
\\\\
Spring è un framework che si basa sul linguaggio Java, rilasciato nel 2003. Ci sono molti progetti scritti
sopra a Spring, come Spring Boot. Spring Boot è il framework che è stato utilizzato per realizzare il vecchio progetto.
\\\\
Spring Boot si occupa di gestire
e fornire tutte le librerie di Spring o di terze parti in base alla configurazione di cui abbiamo bisogno, quindi 
utilizzare questo strumento velocizza notevolmente il processo di sviluppo di un'applicazione.
\\\\
NestJS è un framework basato sul linguaggio JavaScript, in particolare sul linguaggio TypeScript. NestJS è
stato rilasciato nel 2016.
\\\\
Attualmente non sono molti i prodotti realizzati in NestJS, ma questo framework sta prendendo piede negli ultimi anni in quanto
usa Node.js per eseguire il codice JavaScript.
\\\\
Node.js è uno strumento molto utilizzato ma nonostante sia fornito di una vastità di librerie la sua lacuna è la mancanza
di un'architettura ben strutturata, che invece è ben presente in Spring. Quest'architettura 
permette la realizzazione di applicazioni altamente testabili, scalabili, con basso grado di accoppiamento
e facili da manutenere.
\\\\
NestJS copre questa mancanza di Node.js, fornendo un livello in più di astrazione sopra a Node.js e creando un'architettura 
molto simile a quella di Spring.
% //TODO: inserire frase sulla normalizzazione
\section{Punti di valutazione}
Per effettuare l'analisi comparativa tra le due soluzioni, sono stati presi in considerazione i seguenti 
punti e messi a confronto:
\begin{itemize}
    \item facilità di sviluppo;
    \item strumenti di supporto allo sviluppo (documentazione, community, video, ecc..);
    \item facilità di accesso agli strumenti di supporto;
    \item prestazioni;
    \item qualità del codice.
\end{itemize}

\section{Valutazione}
\subsection{Facilità si sviluppo}
\textbf{Spring}
\\\\
Spring Boot si basa sul concetto di Dependency Injection e usa il pattern controller-service-repository. 
Questa struttura permette uno sviluppo rapido delle applicazioni, dato
che si occupa il framework di iniettare le dipendenze dichiarate nei costruttori delle componenti. 
Il pattern controller-service-repository permette di creare applicazioni ben strutturate e dotate di una 
buona separazione delle responsabilità, perseguendo i principi SOLID.
\\\\
\textbf{NestJS}
\\\\
Anche NestJS si basa sul concetto di Dependency Injection e usa il pattern controller-service-repository, per
cui ne trae gli stessi benefici di facilità di sviluppo di Spring.

\subsection{Strumenti di supporto allo sviluppo}
\textbf{Spring}
\\\\
Nel sito ufficiale Spring offre una documentazione molto ricca per imparare in dettaglio il framework.
\\\\
Essendo nato 19 anni fa, negli anni, si è formata una community di supporto molto vasta. 
\\
In questo ventennio sono state pubblicate domande di vario genere alla community, tra cui quelle relative 
ai problemi più comuni che si presentano agli sviluppatori
che affrontano il framework per la prima volta. 
\\
Le risposte degli sviluppatori senior sono molte, ben accurate e scritte in modo chiaro.
\\
Le discussioni sono reperibili nei blog online dedicati o in alcuni siti come Stack Overflow. 
\\\\
Ci sono anche molti video tutorial, reperibili nelle piattaforme più conosciute come il sito Udemy.
\\\\
\textbf{NestJS}
\\\\
Nel sito ufficiale NestJS offre una documentazione molto ricca per imparare in dettaglio il framework.
\\\\
Questo framework è relativamente giovane, nato 6 anni fa. Di conseguenza le aziende e gli sviluppatori che hanno
deciso di adottarlo non sono molti rispetto a quelli che usano Spring.
\\\\
La community è molto piccola.
\\
Le discussioni sui problemi che si presentano agli sviluppatori che affrontano il framework per la prima volta
o sui problemi di carattere generale che si possono presentare sviluppando con questo framework sono molto poche. 
\\\\
Molto spesso se si cerca la soluzione a un problema, non si trova una discussione
a riguardo oppure si trova la discussione aperta da uno sviluppatore ma priva di risposte.
\\\\
A volte invece si viene a scoprire, tramite la discussione ufficiale nel repository GitHub di NestJS, che il 
problema avuto è causato da un bug di NestJS ancora in fase di fix e previsto in rilascio per le versioni
successive.
\\\\
I video tutorial sono molto pochi e non sufficienti per imparare in dettaglio il framework.

\subsection{Facilità di accesso agli strumenti di supporto}
\textbf{Spring}
\\\\
La documentazione di Spring è molto pedante e ricca. Imparare l'uso del framework solo dalla documentazione 
ufficiale richiede molto tempo. 
\\\\
Leggendo interamente la documentazione si diventa dei veri esperti ma con le ridotte tempistiche
dello stage non è possibile effettuare questa cosa.
\\\\
Per iniziare è bene integrare la documentazione ai video tutorial e al materiale online, comprese le discussioni 
della community. 
\\
In questo modo si può essere operativi in tempi più brevi, assicurando di avere una conoscenza
del framework alle spalle sufficiente larga da poter fare le cose in modo coerente.
\\\\
\textbf{NestJS}
\\\\
Lo strumento principale da usare per imparare NestJS è la documentazione ufficiale.
\\\\
La documentazione è fatta molto bene e non è eccessivamente ricca come quella di Spring. 
\\
Spiega in maniera dettagliata
e in modo molto chiaro tutti i concetti di cui si ha bisogno per sviluppare con questo framework.
\\\\
Ho usato principalmente la documentazione ufficiale per imparare NestJS durante lo stage e sono riuscito a essere
operativo in tempi ragionevoli e a fare le cose in modo coerente.

\subsection{Prestazioni}
Per valutare le due soluzioni in termini di prestazioni è stato considerato il tempo medio di risposta delle \gls{api} \gls{rest}, 
caricate sullo stesso 
server ed eseguite a parità di condizioni.
\\\\
Per tutte le \gls{api} \gls{rest} la differenza di risposta tra la soluzione in Spring e in NestJS è troppo piccola per rappresentare un punto
di valutazione significativo (dell'ordine dei millisecondi).
\\\\
Le prestazioni a livello di carico non sono state prese in considerazione, dato che per il momento non si prevede che l'applicazione 
venga utilizzata in modo intensivo.
\subsection{Qualità del codice}
\textbf{Spring}
\\\\
Spring permette di usare degli zuccheri sintattici che nascondono la complessità del codice e lo rendono più
pulito.
\\
Un'esempio sono le annotazioni, che permettono di definire delle proprietà della classe in modo esplicito, chiaro e pulito.
\\
L'annotazione \textit{@Controller} sopra la definizione della
classe, permette di dichiararla come componente Controller.
\\\\
Inoltre già l'architettura interna del framework aiuta a scrivere codice pulito, in quanto la separazione delle
componenti per livello di responsabilità permette di scrivere metodi con poche righe di codice, chiari e
facili da manutenere.
\\\\
\textbf{NestJS}
\\\\
NestJS è dotato degli stessi zuccheri sintattici di Spring per nascondere la complessità del codice e renderlo pulito.
\\\\
La differenza è che su NestJS le annotazioni si chiamano decorator, ma a livello implementativo hanno lo 
stesso significato.
\\\\
Dato che l'architettura di NestJS è la stessa usata da Spring, anche su NestJS si riflettono gli stessi vantaggi di
qualità del codice dovuti all'architettura.
\\\\
Inoltre l'uso dei moduli in NestJS permette di organizzare meglio la struttura dei file rispetto a Spring.
\\\\
TypeOrm, lo strumento ORM di NestJS per interfacciarsi con il database, permette di scrivere query leggermente più
pulite rispetto ad Hibernate, lo strumento ORM di Spring.

\section{Esito}
Da quanto emerso dall'analisi comparativa risulta più semplice creare un software in Spring, per via dell'elevato materiale di
supporto al framework e per la presenza di una community molto attiva e professionale, che negli
anni ha risolto problemi comuni agli sviluppatori che hanno usato il framework. 
\\\\
D'altra parte NestJS risulta povero di materiale di supporto, infatti oltre alla documentazione ufficiale (fatta molto
bene), ne è praticamente sprovvisto e anche la community si può considerare inesistente.
\\\\
Molti dei problemi in NestJS che si presentano agli sviluppatori che affrontano il framework per la prima volta non
trovano soluzione in una community online e questo è fonte di elevato costo in termini di tempo per lo sviluppatore 
inesperto che deve cercare di risolvere il problema da solo.
\\\\
Il fatto di dover risolvere in maniera autonoma il problema, oltre che a portare un costo in termini di tempo, non assicura 
che la soluzione trovata sia una best practice; creando potenziali inconsistenze nel codice.
\\\\
A livello di prestazioni non sono state rilevate sostanziali differenze e nemmeno la possibilità di scrivere query 
leggermente più pulite con TypeOrm rispetto ad Hibernate è stato un punto sufficiente per favorire 
NestJS rispetto a Spring.
\\\\
Il proponente ha considerato importante il punto riguardo la facilità di sviluppo del codice
e la facilità/disponibilità di accesso agli strumenti di supporto.
\\
Il costo in termini di tempo per lo sviluppo di un progetto in NestJS è ancora troppo elevato rispetto a Spring, proprio
per la difficoltà nel reperire materiale di supporto.
\\
Per questo motivo l'analisi comparativa ha portato a preferire l'uso del software scritto in Spring come \gls{back-end-g}
del progetto Smart Parking.
\\\\
La realizzazione del progetto in NestJS è stata comunque molto utile al proponente, che ha maturato competenze riguardo a un nuovo
framework; competenza utile in un futuro prossimo se NestJS prenderà piede e il materiale a 
disposizione e la community cresceranno, rendendo il framework competitivo a livello di Spring.
\\
L'azienda inoltre ha maturato una risorsa interna (me stesso) nel framework NestJS spendibile
per progetti futuri. 