% !TEX encoding = UTF-8
% !TEX TS-program = pdflatex
% !TEX root = ../tesi.tex

%**************************************************************
\chapter{Analisi comparativa}
\label{cap:analisi-comparativa}
%**************************************************************

Il proponente ha deciso di realizzare questo progetto di stage allo scopo di effettuare 
un'analisi comparativa tra la versione del prodotto finale in NestJS e quella in Spring.
\\\\
Essendo NestJS un framework abbastanza giovane (rilasciato nel 2016), il proponente si 
è interessato a questa tecnologia nell'ottica di poter sviluppatore futuri progetti 
aziendali con questo framework e sfruttarne i benefici.
\\\\
E' stato ritenuto troppo rischioso però utilizzare la nuova tecnologia per realizzare
progetti aziendali senza conoscerla nel dettaglio. 
\\
Si è deciso quindi di migrare un progetto già 
esistente scritto in Spring in un progetto con le stesse funzionalità ma scritto in NestJS; con 
lo scopo di approfondire la tecnologia e fare un'analisi comparativa
con il progetto in Spring, che è un framework già ben noto al proponente.
\\\\
Nel caso l'analisi comparativa dia dei buoni punti a favore di NestJS, il proponente può 
decidere di mettere in produzione il progetto realizzato in NestJS, sostituendo quello
in Spring.
\\
In caso contrario verrà mantenuto in produzione il software in Spring, senza gravi perdite in termine 
di costo.
\\\\
Se fosse stato realizzato un progetto in NestJS da zero, senza una controparte in Spring (o un altro framework ben noto al proponente)
e fossero sorti dei problemi di non fattibilità, non preventivati in quanto nella fase di analisi 
non si aveva una conoscenza sufficientemente ampia del framework per poterli rilevare, i
costi per dover riscrivere il progetto sarebbero stati molto più alti.
\\
Avendo invece una controparte funzionante e in ambiente di produzione non si corre questo tipo di rischio.

\section{Le due soluzioni}
Le due soluzioni sviluppano un prodotto con le stesse funzionalità: un set di REST API, per la gestione delle operazioni CRUD del
progetto Smart Parking.
\\\\
Il prodotto già esistente è stato realizzato in Spring, l'altro in NestJS:
\\\\
Spring è un framework che si basa sul linguaggio Java, rilasciato nel 2003. Ci sono molti progetti scritti
sopra a Spring, come Spring Boot. Spring Boot è il framework che è stato utilizzato per creare la controparte 
di questo progetto da un altro stagista.
\\\\
Spring Boot si occupa di gestire
e fornire tutte le librerie di Spring o di terze parti in base alla configurazione di cui abbiamo bisogno, quindi 
utilizzare questo strumento velocizza notevolmente il processo di sviluppo di un'applicazione.
\\\\
NestJS è un framework basato sul linguaggio JavaScript, in particolare sul linguaggio TypeScript. NestJS è
stato rilasciato nel 2016.
\\\\
Attualmente non sono molti i prodotti realizzati in NestJS, ma questo framework sta prendendo piede negli ultimi anni in quanto
usa Node.js per eseguire il codice JavaScript.
\\\\
Node.js è uno strumento molto utilizzato, il suo problema è che la vastità di librerie di cui è fornito non 
risolvono la sua mancanza principale, ovvero la presenza di un'architettura ben strutturata, come invece è ben presente in Spring, che 
permetta la realizzazione di applicazioni altamente testabili, scalabili, con basso grado di accoppiamento
e facili da manutenere.
\\\\
NestJS copre questa mancanza di Node.js, fornendo un livello in più di astrazione sopra a Node.js e creando quindi un'architettura 
molto simile a quella di Spring.

\section{Punti di valutazione}
Per effettuare quest'analisi comparativa tra i 2 prodotti finali, sono stati presi in considerazione i seguenti 
punti:
\begin{itemize}
    \item Facilità di sviluppo.
    \item Strumenti di supporto allo sviluppo (documentazione, community, video, ecc..).
    \item Facilità di accesso agli strumenti di supporto.
    \item Prestazioni.
    \item Qualità del codice.
\end{itemize}

\section{Valutazione}
\subsection{Facilità si sviluppo}
\textbf{Spring}
\\\\
Spring Boot si basa sul concetto di Dependency Injection e usa il pattern controller-service-repository
per sviluppare le applicazioni. Questa struttura permette uno sviluppo rapido dell'applicazione, dato
che si occupa il framework in maniera automatica, di iniettare le dipendenze dichiarate nei costruttori dei componenti e 
il pattern controller-service-repository permette di creare un'applicazione ben strutturata e con una 
buona separazione delle responsabilità, perseguendo i principi SOLID.
\\\\
\textbf{NestJS}
\\\\
Anche NestJS si basa sul concetto di Dependency Injection e usa il pattern controller-service-repository, per
cui ne trae gli stessi benefici di facilità di sviluppo di Spring.

\subsection{Strumenti di supporto allo sviluppo}
\textbf{Spring}
\\\\
Nel suo sito ufficiale Spring offre una documentazione molto ricca per imparare in dettaglio il framework.
\\\\
Essendo questo framework nato 19 anni fa, si è formata una grande community di supporto attorno al framework. 
\\
Infatti negli anni sono state pubblicate domande ai problemi più comuni che si presentano agli sviluppatori
che affrontano questo framework per la prima volta. 
\\
Tali discussioni sono reperibili nei blog appositi o in siti come Stack Overflow. Le risposte alle domande
degli sviluppatori sono tante, molte sono da evitare ma non è difficile imbattersi in risposte
corrette e molto accurate, scritte da sviluppatori con molta esperienza pregressa col framework.
\\\\
Ci sono anche molti video tutorial, in piattaforme come il sito Udemy, per imparare il framework.
\\\\
\textbf{NestJS}
\\\\
NestJS nel suo sito ufficiale offre una documentazione molto ricca per imparare in dettaglio il framework.
\\\\
Questo framework è relativamente giovane, nato 6 anni fa. Di conseguenza le aziende e gli sviluppatori che hanno
deciso di utilizzarlo non sono molte rispetto a quelli che usano Spring e la community è molto piccola.
\\\\
Le discussioni sui problemi che si presentano agli sviluppatori che affrontano il framework per la prima volta
o sui problemi in generale che si possono avere sviluppando con questo framework sono molto poche. 
\\\\
Molto spesso se si cerca la risposta a un problema che si sta avendo con NestJS, non si trova nessuna discussione
a riguardo oppure si trova la domanda posta da uno sviluppatore ma senza alcuna risposta.
\\\\
A volte invece si viene a scoprire, tramite la discussione ufficiale sul repository GitHub di NestJS, che il 
problema che si sta avendo è dovuto a un bug di NestJS ancora in fase di fix e previsto in rilascio per le versioni
successive.
\\\\
I video tutorial sono molto pochi e non sufficienti per imparare in dettaglio il framework.

\subsection{Facilità di accesso agli strumenti di supporto}
\textbf{Spring}
\\\\
La documentazione di Spring è molto pedante e ricca. Imparare l'uso del framework solo dalla documentazione 
ufficiale richiede molto tempo. 
\\\\
Leggendo interamente la documentazione si diventa dei veri esperti del framework ma con le ridotte tempistiche
dello stage non è possibile effettuare questa cosa.
\\\\
Per iniziare è bene integrare la documentazione ai video tutorial e al materiale online, comprese le discussioni 
della community. 
\\
In questo modo si può essere operativi in tempi più brevi, assicurando di avere una conoscenza
del framework alle spalle sufficiente larga da poter fare le cose in modo coerente.
\\\\
\textbf{NestJS}
\\\\
Lo strumento principale da usare per imparare NestJS è la documentazione ufficiale.
\\\\
La documentazione è fatta molto bene e non è eccessivamente ricca come quella di Spring. Spiega in maniera dettagliata
e in modo molto chiaro tutti i concetti di cui si ha bisogno per sviluppare con questo framework.
\\\\
Ho usato principalmente la documentazione ufficiale per imparare il framework durante lo stage e sono riuscito a essere
operativo in tempi ragionevoli e a fare le cose in maniera coerente.

\subsection{Prestazioni}
% //TODO: implementare questa sezione
\subsection{Qualità del codice}
\textbf{Spring}
\\\\
Spring permette di usare degli zuccheri sintattici che nascondono la complessità del codice e lo rendono più
pulito.
\\
Come ad esempio le annotazioni, che permettono in maniera chiara e pulita di definire delle proprietà della classe;
ad esempio di marcarla come Controller semplicemente applicando l'annotazione @Controller sopra la dichiarazione della
classe.
\\\\
Inoltre l'architettura interna del framework permette di per se di scrivere codice pulito, in quanto la separazione delle
componenti per livello di responsabilità permette di scrivere metodi con poche righe di codice, facili da interpretare e
da manutenere.
\\\\
\textbf{NestJS}
\\\\
NestJS è dotato degli stessi zuccheri sintattici per nascondere la complessità del codice e renderlo più pulito.
\\\\
La differenza rispetto a Spring è che su NestJS le annotazioni si chiamano decorator, ma a livello implementativo hanno lo 
stesso significato.
\\\\
Dato che l'architettura di NestJS è la stessa usata da Spring, anche su NestJS si riflettono gli stessi vantaggi di
qualità del codice dovuti all'architettura.
\\\\
L'uso dei moduli in NestJS permette di organizzare meglio la struttura dei file rispetto a Spring.
\\\\
Inoltre TypeOrm, lo strumento ORM di NestJS per interfacciarsi con il database, permette di scrivere query leggermente più
pulite rispetto ad Hibernate, lo strumento ORM di Spring.
% //TODO: Valutare se inserire qui uno script che mostra come NestJS includa le dipendenze nelle query rispetto a Spring

\section{Esito}
Da quanto emerso dall'analisi comparativa risulta più semplice creare un software in Spring, per l'elevato materiale di
supporto al framework presente in rete e per la presenza di una community molto attiva e professionale, che negli
anni ha risolto problemi comuni agli sviluppatori che si sono interfacciati al framework per la prima volta. 
\\\\
Dall'altra parte NestJS risulta povero di materiale di supporto, infatti oltre alla documentazione ufficiale (fatta molto
bene), è praticamente sprovvisto di materiale di supporto e di una community.
\\\\
Infatti molti dei problemi che si presentano agli sviluppatori che affrontano il framework per la prima volta, non sono
stati risolti da una community online e questo è fonte di elevato costo in termini di tempo per lo sviluppatore che con 
una scarsa conoscenza del framework deve cercare di risolvere il problema da solo.
\\\\
Il fatto di risolvere in maniera autonoma il problema, oltre che a portare un costo in termini di tempo, non assicura 
che la soluzione trovata sia il modo corretto di procedere; si vengono quindi a creare potenziali inconsistenze nel codice
scritto.
\\\\
A livello di prestazioni non sono state rilevate sostanziali differenze e nemmeno la possibilità di scrivere query 
leggermente più pulite con TypeOrm rispetto a Spring è stato valutato come parametro sufficiente per scegliere 
NestJS rispetto a Spring.
\\\\
Il proponente ha valutato molto importante il punto di valutazione riguardo la facilità di scrittura del codice
e la facilità/disponibilità di accesso agli strumenti di supporto.
\\
Il costo in termini di tempo per lo sviluppo di un progetto in NestJS è ancora troppo alto rispetto a Spring, proprio
per la difficoltà di reperire materiale di supporto.
\\
Per questo motivo l'analisi comparativa ha portato a preferire l'utilizzo del software scritto in Spring come back-end
di produzione del progetto Smart Parking.
\\\\
La realizzazione del progetto è stata comunque molto utile al proponente, che ha conosciuto un nuovo
framework; il che potrebbe rivelarsi utile in un futuro prossimo se NestJS prenderà piede e quindi il materiale a 
disposizione e la community cresceranno, rendendo il framework competitivo a livello di Spring.