% !TEX encoding = UTF-8
% !TEX TS-program = pdflatex
% !TEX root = ../Volpe_Andrea.tex

%**************************************************************
\chapter{Introduzione}
\label{cap:introduzione}
%**************************************************************

% Introduzione al contesto applicativo.\\

% \noindent Esempio di utilizzo di un termine nel glossario \\
% \gls{api}. \\

% \noindent Esempio di citazione in linea \\
% \cite{site:agile-manifesto}. \\

% \noindent Esempio di citazione nel pie' di pagina \\
% citazione\footcite{womak:lean-thinking} \\

%**************************************************************
\section{L'azienda}
% //TODO: aggiungere parte normalizzazione
Sync Lab nasce a Napoli nel 2002 come software house ed è rapidamente cresciuta nel
mercato dell’Information and Comunications Tecnology (ICT). 
\\\\
A seguito di una
maturazione delle competenze tecnologiche, metodologiche ed applicative nel dominio
del software, l’azienda è riuscita rapidamente a trasformarsi in \gls{System Integrator}\glsfirstoccur conquistando 
significative fette di mercato nei settori mobile, videosorveglianza e sicurezza
delle infrastrutture informatiche aziendali. 
\\\\
Attualmente, Sync Lab ha più di 150 clienti
diretti e finali, con un organico aziendale di 300 dipendenti distribuiti tra le 6 sedi
dislocate in tutta Italia.
Sync Lab si pone come obiettivo principale quello di supportare il cliente nella realizzazione, 
messa in opera e governance di soluzione \gls{IT}\glsfirstoccur, sia dal punto di vista tecnologico,
sia nel governo del cambiamento organizzativo.

\begin{figure}[H]
    \centering
    \includegraphics[height=2.5cm]{logo-synclab}
    \caption{Logo Sync Lab}
\end{figure}

%**************************************************************
\section{Scelta dell'azienda}
Sono venuto a conoscenza dell'azienda Sync Lab grazie al corso d'ingegneria del
software, dove l'azienda è stata il proponente del mio progetto.
\\\\
Sono venuto a conoscenza della proposta di stage di Sync Lab grazie all'evento stage-it 2022. 
L’evento promosso da Assindustria Venetocentro in collaborazione con l’Università 
di Padova per favorire l’incontro tra aziende con progetti innovativi in ambito \gls{IT} e 
studenti dei corsi di laurea in Informatica, Ingegneria informatica e Statistica.
\\\\


%**************************************************************
\section{Introduzione al progetto}

Il progetto di stage consisteva nella migrazione di un progetto esistente realizzato con il framework
Spring, in un progetto con le stesse funzionalità ma realizzato con il framework NestJS.
\\
Il progetto esistente è composto da un set di
\gls{API}\glsfirstoccur \gls{REST}\glsfirstoccur, che vanno a esporre le funzionalità \gls{CRUD}
di un progetto di Smart Parking. 
\\\\
Il proponente ha deciso di effettuare 
la migrazione per fare un'analisi comparativa tra le due soluzioni
e poter valutare quale delle due meglio si adattasse alle esigenze
del prodotto.
\\\\
Il progetto si chiama Smart Parking e consisteva nella realizzazione di una webapp che si occupasse di gestire un sistema
di controllo parcheggi auto in maniera smart. 
\\
Il sistema va ad interrogare una base di dati contenente
l'informazione inerente allo stato di alcuni sensori di parcheggio, fornendo la visualizzazione
dei posti liberi/occupati all'interno di una mappa.
\\\\
L'idea del progetto nasce per agevolare un utente che vuole usufruire di un posto auto all'interno di 
un parcheggio e non vuole perdere tempo in cerca di un posto libero e nemmeno uscire di casa se i posti auto
sono tutti occupati; infatti la webapp oltre a mostrare su una mappa le piazzole libere/occupate, segnala 
anche la disponibilità di posti auto in un parcheggio; il tutto in tempo reale.
\\\\
Era prevista la realizzazione anche di una sezione dedicata ai manutentori, in modo che potessero monitorare
in tempo reale lo stato dei sensori, facilitando quindi il processo di manutenzione.
\\\\
Il progetto prevedeva la realizzazione di un \gls{front-end}\glsfirstoccur con il framework Angular che interagisse 
con le \gls{API} \gls{REST} da realizzare in questo progetto di 
\gls{back-end}\glsfirstoccur.
\clearpage
\begin{figure}[H]
    \centering
    \includegraphics[height=9cm]{front-end-smart-parking}
    \caption{Front-end Smart Parking}
\end{figure}

%**************************************************************
\section{Problematiche riscontrate}
Durante lo svolgimento del progetto si sono presentate alcune criticità, in parte dovute alla mancanza
di conoscenza delle tecnologie da utilizzate. 
\\
Le problematiche riscontrate:

\begin{itemize}
    \item architettura a microservizi: avevo solo una conoscenza basilare
          della tecnologia grazie al corso d'ingegneria del software ma non
          sufficiente per fare un'analisi di migrazione futura del progetto in un'architettura a microservizi;
    \item framework Spring: la conoscenza di questo framework era completamente assente
        ed era importante averla per poter comprendere con chiarezza il software esistente
        di cui doveva essere effettuata la migrazione;
    \item framework Node.js e NestJS: la conoscenza di questi due framework era completamente
        assente. Conoscerli era di fondamentale importanza per poter implementare il servizio
        di \gls{API} \gls{REST} richiesto;
    \item la quantità di \gls{API} \gls{REST} da migrare era troppo elevata per il tempo a disposizione;
    \item il modello della base dati ha dovuto subire adeguamenti rispetto alla prima versione
        per rappresentare nel modo migliore lo scenario funzionale.
\end{itemize}

%**************************************************************
\section{Soluzione scelta}

E' stato scelto di sviluppare il progetto con un architettura di tipo layered architecture. Questo è uno
degli stili architetturali più utilizzati quando si sviluppa un software monolitico. L'idea dietro a 
questa architettura è che i moduli con funzionalità simili sono organizzati in livelli
orizzontali. Quindi ogni livello svolge uno specifico ruolo nell'applicazione.
\\\\
La layered architecture astrae la visione del sistema nel suo insieme, fornendo dettagli 
sufficienti per comprendere ruoli e le responsabilità dei singoli livelli e le relazioni
che intercorrono tra loro.
\\\\
Un'analisi fatta dal proponente ha rivelato questo tipo di architettura adattarsi molto bene al 
servizio di \gls{API} \gls{REST} da realizzare per questo progetto.
\\
Inoltre molti framework per lo sviluppo di applicativi \gls{back-end} si basano su quest'architettura, compresi
Spring e NestJS, che integrano il pattern controller-service-repository; un pattern
che sfrutta la layered architecture, creando tre diversi livelli di astrazione: 
\begin{itemize}
    \item controller: è il livello più alto ed è l'unico responsabile dell'esposizione delle
        funzionalità, in modo che possano essere consumate da entità esterne;
    \item service: livello centrale, gestisce tutta la business logic;
    \item repository: livello più basso, è responsabile di salvare e recuperare i dati da un
        sistema di persistenza, come un database.
\end{itemize}
\clearpage
\leavevmode\newline
Nell'analisi progettuale è stato scelto l'utilizzo della layered architecture come architettura del progetto,
quindi l'utilizzo di framework che integrassero questo tipo di struttura per sviluppare le applicazioni, come 
Spring e NestJS, è stato quasi obbligatorio.
\leavevmode\newline
\begin{figure}[H]
    \centering
    \includegraphics[height=9cm]{controller-service-repository-pattern}
    \caption{Controller-service-repository pattern}
\end{figure}
\leavevmode\newline
Non era prevista la creazione di un sistema di autenticazione per l'accesso alle \gls{API} \gls{REST}, in quanto 
un altro studente tirocinante stava realizzando questa parte.

%**************************************************************
\section{Descrizione del prodotto ottenuto}

All'inizio dello stage era disponibile un \gls{back-end} contenente le \gls{API} \gls{REST}, sviluppato in Spring, pronto per
poter essere messo in esercizio in ambiente di produzione.
\\
Non potendo migrare l'intero set di \gls{API} \gls{REST}, è stato preventivato di sviluppare
le più importanti, in modo da poter effettuare le operazioni \gls{CRUD}\glsfirstoccur più
comuni.
\\\\
Le \gls{API} \gls{REST} espongono un'interfaccia compatibile con quello che ormai è uno standard per la
comunicazione con servizi di tipo \gls{REST}; ovvero per comunicare con loro è necessario effettuare
delle richieste \gls{HTTP}\glsfirstoccur a degli specifici \gls{endpoint} tramite i seguenti metodi:
\begin{itemize}
    \item GET: per ottenere delle risorse dal servizio \gls{REST};
    \item POST: per creare una nuova risorsa nel servizio \gls{REST};
    \item PUT: per modificare una risorsa nel servizio \gls{REST};
    \item DELETE: per eliminare una risorsa dal servizio \gls{REST}.
\end{itemize}
\leavevmode\newline
Nel \gls{back-end} è presente inoltre un servizio schedulato che ogni due minuti in maniera autonoma va a fare il polling
da un file \gls{XML}\glsfirstoccur online, contenente gli stati aggiornati dei sensori. Questo servizio registra poi 
le variazioni, rispetto
al polling precedente, nello strato di persistenza.
\\\\
Il file \gls{XML} viene scritto e gestito dai produttori dei sensori di parcheggio, quindi non era compito di questo 
progetto gestirne il funzionamento. Il funzionamento di questo file \gls{XML} è comunque abbastanza banale,
in quanto ad ogni variazione di stato, il sensore di parcheggio va semplicemente ad aggiornare 
il record a lui associato
all'interno del file.
\\
\begin{figure}[H]
    \centering
    \includegraphics[height=8cm]{polling-sensori}
    \caption{Polling sensori}
\end{figure}
\leavevmode\newline
%**************************************************************
\section{Tecnologie utilizzate}

\textbf{Git}
\\\\
E' uno degli strumenti di controllo di versionamento più utilizzati. Facilita la collaborazione
tra gli sviluppatori nella realizzazione di un progetto e permette con semplicità di spostarsi
tra varie versioni del software realizzate. Nel progetto è stato utilizzato con il workflow
Gitflow.
\\\\\\
\textbf{Visual Studio Code}
\\\\
E' un editor di codice sorgente sviluppato da Microsoft che aiuta lo sviluppatore durante la fase
di sviluppo del codice in quanto evidenzia le parole chiave, segnala errori di scrittura, suggerisce
snippet di codice. Possiede una grande libreria di estensioni facilmente installabili per renderlo
compatibile con praticamente qualsiasi linguaggio di programmazione.
\\\\\\
\textbf{Postman}
\\\\
E' un'applicazione che viene utilizzata solitamente per testare \gls{API}. E' un client \gls{HTTP} che testa richieste
\gls{HTTP} utilizzando una \gls{GUI}\glsfirstoccur, attraverso la quale otteniamo diversi tipi di risposta in base alle \gls{API} che 
andiamo ad interrogare.
\\\\\\
\clearpage
\leavevmode\newline
\textbf{Stoplight}
\\\\
E' una piattaforma per progettare \gls{API}. Grazie a questo strumento è possibile documentare in maniera rigorosa
e su uno spazio in cloud un set di \gls{API}. La piattaforma permette di specificare varie informazioni per ogni
\gls{API}, tra cui \gls{endpoint}\glsfirstoccur, parametri in ingresso attesi, possibili risposte con status code associato. Questo
strumento è molto utile per gli sviluppatori \gls{front-end} che devono chiamare le \gls{API} di un servizio
\gls{back-end}, soprattutto grazie alla funzionalità che permette di generare il \gls{mock}\glsfirstoccur della risposta di un'\gls{API}, 
permettendo allo sviluppatore di effettuare le chiamate al \gls{back-end} anche senza che questo sia stato ancora realizzato.
\\\\\\
\textbf{TypeScript}
\\\\
E' un superset di JavaScript, che aggiunge tipi, classi, interfacce e moduli opzionali al JavaScript 
tradizionale. Si tratta sostanzialmente di una estensione di JavaScript.
TypeScript è un linguaggio tipizzato, ovvero aggiunge definizioni di tipo statico: i tipi consentono di 
descrivere la forma di un oggetto, documentandolo meglio e consentendo a TypeScript di verificare che 
il codice funzioni correttamente.
\\\\\\
\textbf{Node.js}
\\\\
E' un runtime system open source per eseguire applicazioni scritte in JavaScript, permettendoci di utilizzare questo 
linguaggio, tipicamente utilizzato nella client-side, anche per la scrittura di applicazioni server-side.
La piattaforma è basata sul JavaScript Engine V8, che è il runtime di Google utilizzato anche da Chrome e 
disponibile sulle principali piattaforme, anche se maggiormente performante su sistemi operativi UNIX-like.
\\\\\\
\textbf{NestJS}
\\\\
E' un framework per la creazione di applicazioni lato server Node.js efficienti e scalabili. 
Utilizza JavaScript ma è costruito con e supporta completamente TypeScript. Aggiunge un livello di astrazione
al framework Express, che a sua volta aggiunge astrazione al framework Node.js. Di conseguenza NestJS 
utilizza Node.js per eseguire il codice JavaScript generato dalla compilazione del codice TypeScript.
\\\\\\
\textbf{Spring}
\\\\
Spring è un framework leggero, basato su Java. Questo framework integra soluzioni a vari problemi tecnici
che si presentano con alta frequenza durante lo sviluppo software. Spring si basa su due design pattern
fondamentali che sono l'Inversion of Control e Dependency Injection.
\\\\\\
\clearpage
\leavevmode\newline
\textbf{PostgreSQL}
\\\\
Chiamato anche Postgres, è un sistema di database relazionale a oggetti (ORDBMS), open source e 
gratuito.
Le principali caratteristiche di Postgres sono affidabilità, integrità dei dati, funzionalità ed estensibilità, 
oltre alla propria community open source che gestisce, aggiorna e sviluppa soluzioni performanti e innovative.
\\\\\\
\textbf{Jest}
\\\\
Jest è un framework di unit test sviluppato da Facebook. Focalizzato sulla semplicità, è utilizzabile in qualsiasi
progetto JavaScript. E'uno dei framework di test JavaScript più popolare in questi giorni e la scelta di default 
per alcuni framework come NestJS e React.
\\\\\\
\textbf{Winston}
\\\\
Winston è una delle libreria più famose per effettuare il logging su applicazioni Node.js. Permette di effettuare il logging
su più livelli di informazione, formattare il logging in modo predefinito, scegliere la destinazione di output del log e molte 
altre opzioni.
\\\\\\
\textbf{Npm}
\\\\
E' uno dei gestori di pacchetti per il linguaggio JavaScript più popolare. E' il gestore di pacchetti predefinito 
per Node.js.

%**************************************************************
\section{Organizzazione del testo}

\begin{description}
    \item[{\hyperref[cap:analisi-requisiti]{Il secondo capitolo}}] descrive l'analisi dei requisiti.
    
    \item[{\hyperref[cap:progettazione]{Il terzo capitolo}}] approfondisce la fase di progettazione.
    
    \item[{\hyperref[cap:ristrutturazione-database]{Il quarto capitolo}}] descrive la fase di ristrutturazione del database.
    
    \item[{\hyperref[cap:verifica-e-validazione]{Il quinto capitolo}}] descrive la fase di verifica e validazione.
    
    \item[{\hyperref[cap:analisi-comparativa]{Il sesto capitolo}}] approfondisce l'analisi comparativa tra la soluzione in Spring e quella in NestJS.
    
    \item[{\hyperref[cap:conclusioni]{Il settimo capitolo}}] presenta le conclusioni finali sul progetto e sull'esperienza di stage.
\end{description}

Riguardo la stesura del testo, relativamente al documento sono state adottate le seguenti convenzioni tipografiche:
\begin{itemize}
	\item gli acronimi, le abbreviazioni e i termini ambigui o di uso non comune menzionati vengono definiti nel glossario, situato alla fine del presente documento;
	\item per la prima occorrenza dei termini riportati nel glossario viene utilizzata la seguente nomenclatura: \emph{parola}\glsfirstoccur.
\end{itemize}