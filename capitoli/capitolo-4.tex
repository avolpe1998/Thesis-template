% !TEX encoding = UTF-8
% !TEX TS-program = pdflatex
% !TEX root = ../tesi.tex

%**************************************************************
\chapter{Verifica e validazione}
\label{cap:verifica-e-validazione}
%**************************************************************

\section{Verifica}
\subsection{Criteri di verifica}
E' stato stabilito con il proponente che il prodotto finale dovesse avere 
una suite di test automatici per testare la business logic del progetto
con una copertura a livello di branch stabilita >= al 90\% e una copertura
a livello linee di codice >= al 60\%.
\\\\
La motivazione che ha portato a richiedere una copertura così alta a livello
branch è dovuta al fatto che si vuole assicurare il corretto funzionamento di 
tutti i possibili casi di errore anche in caso di futura manutenzione del software.
\\\\
E' inoltre importante che sia mantenuta la coerenza stabilita per i casi di errore e
nel caso alcuni branch non fossero coperti da casi di test, uno sviluppatore potrebbe
inconsciamente modificare il comportamento del programma in caso di un errore, generando
un comportamento inaspettato del programma per l'utente finale. 
\\\\
E' molto importante per la proponente che questa cosa non succeda e quindi è stata richiesta
una copertura a livello branch >= al 90\%.

\subsection{Strumenti utilizzati}
NestJS fornisce al suo interno uno strumento per scrivere vari tipi di test, tra cui unit test, 
end to end test, integration test e così via. 
\\\\
Lo strumento di test usa Jest, un framework apposito per scrivere test automatici ed effettuare 
il mock dei componenti. Jest funziona su progetti che includono  React, Babel, TypeScript, Node,
Angular, Vue.
\\\\
I file test in NestJS per conformità sono stati nominati con lo stesso nome della classe che vanno
a testare e devono terminare in .spec.ts se vogliamo che vengano visti da NestJS ed eseguiti.
\\\\
Ogni caso di test deve essere racchiuso all'interno della dicitura describe, specificando nome della
classe che si va a testare e una funzione di callback contenente i test per i vari metodi.
\\
A sua volta ogni specifico metodo che si va a testare deve essere racchiuso in un describe che deve
contenere il nome del metodo e una funzione callback contenente tutti i test su quel metodo.
\\
\begin{lstlisting}
describe('MaintainersRegistryService', () => {
  describe('getMaintainerById', () => {
    it('should return the maintainer if found', async () => {
      jest.spyOn(maintainersRegistryRepository, 'findOne')
        .mockImplementation(() => Promise.resolve(maintainerRegistry));

      const response = maintainersRegistryService.getMaintainerById('1');

      await expect(response).resolves.toEqual(maintainerRegistry);
    });

    it('should thrown a NotFoundError if the maintainer was not found', async () => {
      jest.spyOn(maintainersRegistryRepository, 'findOne')
        .mockImplementation(() => Promise.resolve(null));

      const response = maintainersRegistryService.getMaintainerById('1');

      await expect(response).rejects.toThrow(NotFoundError);
    });
  });
\end{lstlisting}

\subsection{Progettazione}
Per i criteri di verifica stabiliti con la proponente si è deciso di testare tutti i metodi dei service che
avessero un numero di branch > 1. In questo modo viene assicurata la copertura di tutti i casi di errore
dato che i metodi con un solo branch (solitamente metodi di get delle informazioni di un'entità) in caso 
di errore non contengono logica di business per lanciare errori custom ma al massimo lanciano eccezioni
TypeORM o NestJS catturate dall'exception filter che conforma la risposta per il client.
\\\\
Scrivendo dei test che vanno a controllare i metodo con più branch si assicura che:
\begin{itemize}
    \item ogni branch può essere raggiunto
    \item ogni branch si comporta come aspettato
    \item se uno sviluppatore modifica un metodo, si assicura che i casi di errore (id passato non trovato, 
    aggiornamento non riuscito ecc..) non vengano eliminati e si assicura il mantenimento della loro
    conformità con quanto stabilito con la proponente. 
    \\
    Ad esempio si assicura che venga mantenuto il controllo che in caso di errore di aggiornamento di un'entità
    venga lanciata, dal service, un eccezione di tipo UpdateError.
\end{itemize}
\leavevmode\newline
Per alleggerire i test e quindi velocizzarne l'esecuzione si è deciso di creare dei mock per tutte le dipendenze
della classe da testare, repository compresi.
\\\\
Fornire dei mock per le dipendenze significa fornire ai casi di test delle componenti fittizie che vanno a simulare
il comportamento della componente originale. 
\\\\
Le componenti fittizie sono molto più piccole delle componenti originali e implementano solo la parte di codice 
necessaria a far funzionare il caso di test nel modo in cui ci si aspetta.
\\\\
Jest fornisce dei metodi molto utili per creare il mock di un metodo di una classe e sovrascriverne il comportamento.
Tramite il metodo spyOn infatti si va a specificare l'oggetto e il suo metodo di cui si vuole effettuare il mock.
\\\\
Per evitare di dover caricare componenti reali, quindi molto pesanti ed effettuare il mock dei loro metodi, è stata
usata una libreria esterna chiamata @golevelup/ts-jest, che permette di dichiarare l'intera classe di cui si vuole fare
il mock e restituisce un oggetto con gli stessi metodi della classe che gli abbiamo specificato ma con il contenuto 
dei metodi vuoto, 
alleggerendo quindi notevolmente il peso della componente.
\\\\
Di queste componenti viene poi effettuato il mock dei metodi necessari all'esecuzione dei test tramite il metodo
spyOn, come descritto in precedenza.

\subsection{Realizzazione}
Come risultato di quanto progettato sono stata creata una suite di 16 casi di test, per un totale di 59 unit test.
\\
\begin{table}
    \begin{tabular}{|p{5cm}|p{1.4cm}|p{1.5cm}|p{1.4cm}| p{1.4cm} |} 
    \hline
    File & \% Stmts & \% Branch &  \% Funcs & \% Lines \\ 
    \hline
    maintainers-registry.service.ts & 91.66 & 100 & 83.33 & 91.17 \\ 
    \hline
    parking-areas.service.ts & 82.14 & 100 & 46.15 & 72.13 \\ 
    \hline
    parking-sensors.service.ts & 94.11 & 100 & 94.11 & 92.3 \\ 
    \hline
    parking-spots.service.ts & 88.88 & 100 & 50 & 87.87 \\ 
    \hline
    sensors.service.ts & 38.46 & 100 & 0 & 27.27 \\ 
    \hline
    sensors-maintenance.service.ts & 86.11 & 90 & 71.42 & 84.84 \\ 
    \hline
    sensors-scraping.service.ts & 65.95 & 100 & 36.36 & 51.78 \\ 
    \hline
    maintainers-registry.service.ts & 91.66 & 100 & 83.33 & 91.17 \\ 
    \hline
    \end{tabular}
\end{table}

\subsection{Scheduler}
Si è deciso di implementare il polling dei dati del sensore dal file XML online
tramite la libreria @nestjs/schedule che integra il package cron di Node.js.
\\\\
Questa libreria mette a disposizione uno strumento per poter eseguire il metodo
di una classe ad intervalli di tempo regolari.
\\\\
Per fare ciò bisogna importare nel modulo root, il modulo della libreria schedule in questo modo:
\begin{lstlisting}
@Module({
  imports: [
    ScheduleModule.forRoot()
  ],
})
export class AppModule {}
\end{lstlisting}
\leavevmode\newline
Successivamente abbiamo a disposizione un decorator @Cron, da indicare sopra al metodo che vogliamo venga 
schedulato. 
\\\\
Come parametro del @Cron dobbiamo inserire la stringa, rispettando il cron pattern, che indica
l'intervallo di tempo con cui vogliamo che NestJS esegua il metodo.
\\\\
Nel nostro caso vogliamo eseguirlo ogni due minuti quindi gli passiamo la stringa "*/2 * * * *".
\\\\