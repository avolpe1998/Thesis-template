% !TEX encoding = UTF-8
% !TEX TS-program = pdflatex
% !TEX root = ../tesi.tex

%**************************************************************
\chapter{Ristrutturazione database}
\label{cap:ristrutturazione-database}
%**************************************************************

Durante la fase di studio del database mi sono reso conto che aveva dei problemi, che hanno
richiesto una sua ristrutturazione.
\\\\
Mi sono confrontato poi con un mio collega stagista che stava lavorando sullo stesso database, per
un altro progetto di stage collegato a Smart Parking.
\\
Anche lui era d'accordo a ristrutturare il database e mi ha dato un aiuto a rilevare altri possibili
problemi durante la fase di progettazione della ristrutturazione del database.
\\\\
Abbiamo poi presentato il progetto di ristrutturazione al proponente che si è mostrato d'accordo con
noi sull'esistenza dei problemi rilevati ed è stato molto soddisfatto della soluzione proposta.
\\\\
Il modello logico esistente era il seguente:
\begin{figure}[!h]
  \centering
  \includegraphics[height=9cm]{modello-logico-vecchio}
\end{figure}

\section{Problemi rilevati}
Nomi tabelle incoerenti
\\\\
I nomi di alcune tabelle erano incoerenti con la funzionalità che andavano a svolgere.
\\
sensors\_maintainer è la tabella che contiene i dati di manutenzione dei sensori, un 
nome più appropriato potrebbe essere sensors\_maintenance.
\\\\
parking\_area è la tabella che rappresenta la piazzola di parcheggio, un nome più appropriato
potrebbe essere parking\_spots.
\\\\
Duplicazione di dati
\\\\
La tabella parking\_area\_stats non serve a nulla, in quanto duplica soltanto dei dati già presenti
nella tabella parking\_area, creando un'inutile ridondanza di dati.
\\\\
Cardinalità delle relazioni errata
\\\\
La relazione sensors -> parking\_area ha cardinalità uno a molti, nel senso che una piazzola può avere
più sensori e un sensore può essere associato solo a una piazzola. 
\\
La cosa è sbagliata, in quanto una piazzola può avere più sensori associati (un sensore di parcheggio e/o N
sensori ambientali) ma anche un sensore può essere associato a più piazzole; in quanto un sensore ambientale
può ricoprire un area di N piazzole.
\\\\
Mancanza di tabelle fondamentali
\\\\
Manca una tabella fondamentale che rappresenti un parcheggio (un insieme di piazzole). 
\\
Tabella molto importante dato che essa ha un indirizzo e una posizione geografica (latitudine e longitudine), 
riconosciute come tali dagli strumenti di navigazione più comuni, come Google Maps. 
\\
Ogni piazzola di uno stesso parcheggio hanno una posizione geografica, diversa dalle altre piazzole (discostata di qualche 
metro) e potenzialmente diversa da quella del parcheggio. 
\\
Quindi con la struttura vecchia non è possibile ricercare un parcheggio a sistema, passando le coordinate rilevate da Google
Maps ad esempio e nemmeno identificare un parcheggio.
\\\\
Database poco modulare
\\\\
Il valore del sensore viene salvato all'interno della tabella parking\_area. Questa cosa non crea problemi ora che 
si è deciso, dall'analisi fatta, di salvare solo l'ultima misurazione di un sensore di parcheggio.
\\\\
Se però in futuro si decide di salvare uno storico di misurazioni dei sensori di parcheggio (cosa molto probabile che
avvenga e cosa che viene già fatta con i sensori ambientali), con la struttura attuale non è possibile farlo e i costi
per modificare la struttura del database in futuro, con il progetto in produzione da tempo, sarebbero molto più
alti rispetto a farlo adesso.
\\\\
Inoltre l'aggiunta di una tabella per lo storico non ha un impatto negativo sulla struttura del database.

\section{Ristrutturazione}

Il modello logico ristrutturato è il seguente:
\begin{figure}[!h]
  \centering
  \includegraphics[height=9cm]{modello-logico-ristrutturato}
\end{figure}
\\\\
Operazioni effettuate
\\\\
Sono stati modificati i nomi di alcune tabelle per renderle più coerenti alla loro funzionalità:
\\\\
parking\_area è stata modificata in parking\_spots.
\\
sensors\_maintainer è stata modificata in sensors\_maintenance.
\\\\
E' stata eliminata la tabella parking\_area\_stats.
\\\\
Sono state modificate le cardinalità di alcune relazioni:
\\\\
La relazione sensors -> parking\_area è diventata una relazione molti a molti.
\\
E' stato reso facoltativo il fatto che il sensore debba essere associato a una piazzola.
\\
E' stato reso facoltativo il fatto che il manutentore debba essere associato a un sensore.
\\
E' stato reso facoltativo il fatto che il sensore debba essere associato a un manutentore.
\\\\
E' stata aggiunta la tabella parking\_areas che rappresenta i parcheggi, dotati di indirizzo, latitudine e 
longitudine.
\\\\
E' stata creata la tabella parking\_sensors per salvare la misurazione del sensore di parcheggio ed è
stato tolto il campo value dalla nuova tabella parking\_spots; in quanto avendo la tabella parking\_sensors,
il campo value rappresentava una ridondanza inutile.
\\\\
Per ogni sensore di parcheggio ora viene salvata solo una misurazione nella tabella parking\_sensors, ovvero
l'ultima, che sovrascrive la precedente ma a differenza del modello logico precedente, se in qualsiasi momento
si decide di salvare lo storico delle misurazioni dei sensori di parcheggio, si può fare semplicemente togliendo
il vincolo unique dalla foreign key fk\_sensor\_id, senza modificare la struttura delle tabelle.