% !TEX encoding = UTF-8
% !TEX TS-program = pdflatex
% !TEX root = ../Volpe_Andrea.tex

%**************************************************************
\chapter{Conclusioni}
\label{cap:conclusioni}
%**************************************************************

\intro{In questo capitolo vengono tratte le conclusioni finali sulla
realizzazione del progetto.}

\section{Analisi del prodotto ottenuto}
Il prodotto ottenuto anche se pronto per il rilascio in ambiente di produzione, come spiegato nell'analisi
comparativa, non verrà utilizzato.
\\\\
Questa soluzione verrà conservata dal proponente nel proprio repository GitHub, come strumento
da cui prendere spunto per eventuali progetti futuri in NestJS.
\\\\
Il progetto realizzato ha seguito tutte le buone pratiche di sviluppo suggerite nella documentazione ufficiale
NestJS e vari design pattern.
\\\\
Come richiesto dal proponente è stata implementata una suite di unit test automatici per testare tutti i servizi 
a livello branch.
\\\\
Il prodotto realizzato è facile da avviare; qualsiasi sviluppatore inesperto sulla tecnologia NestJS, 
leggendo il contenuto file README sul repository GitHub, è in grado di avviarlo in pochi passi. Il piccolo manuale spiega in modo 
dettagliato come scaricare le 
dipendenze richieste tramite npm e far partite il servizio.

\section{Raggiungimento degli obiettivi}
Vediamo gli obiettivi pianificati e il loro stato di raggiungimento al termine dello stage: 
\\\\
\textbf{Notazione utilizzata:}
\\
\begin{itemize}
    \item O per gli obiettivi obbligatori, vincolanti in quanto obiettivo primario richiesto dal proponente;
    \item D per gli obiettivi desiderabili, non vincolanti o strettamente necessari ma dal riconoscibile valore aggiunto;
    \item F per i vincoli facoltativi, rappresentanti valore aggiunto non strettamente competitivo.
\end{itemize}
\leavevmode\newline
\begin{table}[H]
    \begin{tabular}{|p{1.5cm}|p{7.7cm}|p{2cm}|} 
    \hline
    \textbf{Codice} & \textbf{Descrizione} & \textbf{Stato} \\ 
    \hline
    O01 & Acquisizione competenze Java & Soddisfatto \\
    \hline
    O02 & Acquisizione competenze passaggio da monolite a microservizi & Soddisfatto \\
    \hline
    O03 & Acquisizione competenze framework Spring Boot & Soddisfatto \\
    \hline
    O04 & Acquisizione competenze strumento Spring Data JPA & Soddisfatto \\
    \hline
    O05 & Acquisizione competenze realizzazione servizio di \gls{api} \gls{rest} & Soddisfatto \\
    \hline
    O06 & Acquisizione competenze strumento Node.js & Soddisfatto \\
    \hline
    O07 & Acquisizione competenze framework Express.js & Soddisfatto \\
    \hline
    O08 & Realizzazione di un servizio \gls{rest} prototipale con Spring Boot & Soddisfatto \\
    \hline
    O09 & Realizzazione di un servizio \gls{rest} prototipale con Express.js & Soddisfatto \\
    \hline
    O10 & Realizzazione di un'analisi per il progetto Smart Parking & Soddisfatto \\
    \hline
    O11 & Realizzazione di una progettazione per il progetto Smart Parking & Soddisfatto \\
    \hline
    O12 & Realizzazione di una suite di test automatici con copertura a livello branch >= 90\% & Soddisfatto \\
    \hline
    O13 & Realizzazione di una suite di test automatici con copertura a livello linee di codice >= 60\% & Soddisfatto \\
    \hline
    O14 & Realizzazione di un numero di \gls{api} \gls{rest} progettate maggiore dell'80\% & Soddisfatto \\
    \hline
    O15 & Realizzazione di un'analisi comparativa tra la soluzione in Spring e in NestJS & Soddisfatto \\
    \hline
    O16 & Raggiungimento degli obiettivi richiesti in autonomia seguendo il cronoprogramma & Soddisfatto \\
    \hline
    \end{tabular}
    \caption{Stato raggiungimento obiettivi obbligatori}
\end{table}
\clearpage
\begin{table}[H]
    \begin{tabular}{|p{1.5cm}|p{7.7cm}|p{2cm}|} 
    \hline
    \textbf{Codice} & \textbf{Descrizione} & \textbf{Stato} \\ 
    \hline
    D01 & Realizzazione di un numero di \gls{api} \gls{rest} progettate pari al 100\%  & Soddisfatto \\
    \hline
    \end{tabular}
    \caption{Stato raggiungimento obiettivi desiderabili}
\end{table}
\begin{table}[H]
    \begin{tabular}{|p{1.5cm}|p{7.7cm}|p{2cm}|} 
    \hline
    \textbf{Codice} & \textbf{Descrizione} & \textbf{Stato} \\ 
    \hline
    F01 & Analizzazione di come poter implementare le best practice dell'architettura a microservizi con NestJS & Soddisfatto parzialmente \\
    \hline
    \end{tabular}
    \caption{Stato raggiungimento obiettivi facoltativi}
\end{table}

\section{Valutazione personale}
Ho ritenuto l'esperienza di stage molto positiva. La prima parte dello stage è stata quella più difficile, 
in quanto ho dovuto imparare da zero delle tecnologie che non avevo mai visto come Spring e NestJS.
\\\\
Questo processo di apprendimento è stato per me molto importante, sia perché la conoscenza di questi due framework mi sarà
utile in ambito lavorativo, sia perché mi ha insegnato ad apprendere il funzionamento di una nuova tecnologia in modo
dettagliato e allo stesso tempo ad essere operativo in tempo breve.
\\\\
Ho potuto affacciarmi con un'azienda con molti sviluppatori e quindi confrontarmi giornalmente con persone più 
esperte di me che mi hanno aiutato molto.
\\\\
Sono soddisfatto del prodotto realizzato e di come si sono svolti i processi di sviluppo che hanno portato alla sua realizzazione. 
\\\\
Ritengo molto valida questo tipo di esperienza, anche per tutti i miei colleghi universitari futuri, in quanto mi ha 
fatto crescere molto come persona e sviluppatore. 
\\
Anche se due mesi sono pochi ho potuto mettere mano a un progetto reale che mi ha insegnato molto riguardo le dinamiche 
di sviluppo aziendali.
\\\\
Sicuramente il background fornito dall'università è stato fondamentale per poter svolgere questo stage; in quanto l'università mi ha 
fornito tutti i concetti basilari di cui avevo bisogno per poter apprendere le nuove tecnologie e per poterle utilizzare in maniera 
corretta.
\\\\
L'università mi ha dato gli strumenti per poter analizzare con occhio critico le scelte progettuali che sono andato a fare.
\\\\
In conclusione sono pienamente soddisfatto dello stage fatto e del prodotto realizzato che, anche se non verrà 
utilizzato, ha posto le basi per un futuro progetto realizzabile in NestJS e ha reso appetibile questa tecnologia
per i futuri progetti dell'azienda proponente.