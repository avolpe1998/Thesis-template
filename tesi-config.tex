%**************************************************************
% file contenente le impostazioni della tesi
%**************************************************************

%**************************************************************
% Frontespizio
%**************************************************************

% Autore
\newcommand{\myName}{Andrea Volpe}                                    
\newcommand{\myTitle}{Migrazione e analisi comparativa di un back-end per un servizio di smart parking}

% Tipo di tesi                   
\newcommand{\myDegree}{Tesi di laurea}

% Università             
\newcommand{\myUni}{Università degli Studi di Padova}

% Facoltà       
\newcommand{\myFaculty}{Corso di Laurea in Informatica}

% Dipartimento
\newcommand{\myDepartment}{Dipartimento di Matematica "Tullio Levi-Civita"}

% Titolo del relatore
\newcommand{\profTitle}{Prof.}

% Relatore
\newcommand{\myProf}{Paolo Baldan}

% Luogo
\newcommand{\myLocation}{Padova}

% Anno accademico
\newcommand{\myAA}{2021-2022}

% Data discussione
\newcommand{\myTime}{Dicembre 2022}


%**************************************************************
% Impostazioni di impaginazione
% see: http://wwwcdf.pd.infn.it/AppuntiLinux/a2547.htm
%**************************************************************

\setlength{\parindent}{14pt}   % larghezza rientro della prima riga
\setlength{\parskip}{0pt}   % distanza tra i paragrafi


%**************************************************************
% Impostazioni di biblatex
%**************************************************************
\bibliography{bibliografia} % database di biblatex 

\defbibheading{bibliography} {
    \cleardoublepage
    \phantomsection 
    \addcontentsline{toc}{chapter}{\bibname}
    \chapter*{\bibname\markboth{\bibname}{\bibname}}
}

\setlength\bibitemsep{1.5\itemsep} % spazio tra entry

\DeclareBibliographyCategory{opere}
\DeclareBibliographyCategory{web}

\addtocategory{opere}{womak:lean-thinking}
\addtocategory{web}{site:agile-manifesto}

\defbibheading{opere}{\section*{Riferimenti bibliografici}}
\defbibheading{web}{\section*{Siti Web consultati}}


%**************************************************************
% Impostazioni di caption
%**************************************************************
\captionsetup{
    tableposition=top,
    figureposition=bottom,
    font=small,
    format=hang,
    labelfont=bf
}

%**************************************************************
% Impostazioni di glossaries
%**************************************************************
\makeglossaries

%**************************************************************
% Acronimi
%**************************************************************
\newacronym[description={\glslink{it}{Information Technology}}]
    {IT}{IT}{Information Technology}

\newacronym[description={\glslink{api}{Application Programming Interface}}]
    {API}{API}{Application Programming Interface}

\newacronym[description={\glslink{rest}{Representational State Transfer}}]
    {REST}{REST}{Representational State Transfer}

\newacronym[description={\glslink{crud}{Create Read Update Delete}}]
    {CRUD}{CRUD}{Create Read Update Delete}

\newacronym[description={\glslink{http}{Hypertext Transfer Protocol}}]
    {HTTP}{HTTP}{Hypertext Transfer Protocol}

\newacronym[description={\glslink{xml}{eXtensible Markup Language}}]
    {XML}{XML}{eXtensible Markup Language}

\newacronym[description={\glslink{gui}{Graphical User Interface}}]
    {GUI}{GUI}{Graphical User Interface}

\newacronym[description={\glslink{gps}{Global Positioning System}}]
    {GPS}{GPS}{Global Positioning System}

\newacronym[description={\glslink{orm}{Object Relational Mapping}}]
    {ORM}{ORM}{Object Relational Mapping}

\newacronym[description={\glslink{oop}{Object-Oriented Programming}}]
    {OOP}{OOP}{Object-Oriented Programming}

\newacronym[description={\glslink{dto}{Data Transfer Object}}]
    {DTO}{DTO}{Data Transfer Object}



%**************************************************************
% Glossario
%**************************************************************
\newglossaryentry{System Integrator}
{
    name=\glslink{SYSTEM INTEGRATOR}{system integrator},
    text=System Integrator,
    sort=system integrator,
    description={Il System Integrator si occupa principalmente di integrare sistemi informatici, anche molto eterogenei tra loro, al fine di creare un ambiente informatico che sia unico, funzionale e adatto al tipo di azienda di riferimento. Il suo principale compito consiste dunque nel far dialogare correttamente tra loro diversi apparati - quali tecnologie, software e hardware, cioè componenti virtuali e componenti fisiche di un sistema - al fine di garantire la business continuity in azienda}
}

\newglossaryentry{it}
{
    name=\glslink{it}{IT},
    text=IT,
    sort=it,
    description={L’espressione tecnologia dell’informazione (in inglese information technology, in acronimo IT), indica l’utilizzo di elaboratori e attrezzature di telecomunicazione per memorizzare, recuperare, trasmettere e manipolare dati}
}

\newglossaryentry{api}
{
    name=\glslink{api}{API},
    text=API,
    sort=api,
    description={In un programma informatico indica un insieme di procedure (in genere raggruppate per strumenti specifici) atte a risolvere uno specifico problema di comunicazione tra diversi computer o tra diversi software o tra diversi componenti di software}
}

\newglossaryentry{rest}
{
    name=\glslink{rest}{REST},
    text=REST,
    sort=rest,
    description={E' uno stile architetturale per sistemi distribuiti. Il termine REST rappresenta un sistema di trasmissione di dati su HTTP senza ulteriori livelli. I sistemi REST non prevedono il concetto di sessione, ovvero sono stateless. Il funzionamento prevede una struttura degli URL ben definita che identifica univocamente una risorsa o un insieme di risorse e l'utilizzo dei metodi HTTP specifici per il recupero di informazioni (GET), per la modifica (POST, PUT, PATCH, DELETE) e per altri scopi (OPTIONS, ecc.)}
}

\newglossaryentry{back-end}
{
    name=\glslink{BACK-END}{back-end},
    text=back-end,
    sort=back-end,
    description={E' la parte non visiva all’utente che approda sulla piattaforma. Questa, infatti, è destinata alla gestione del sito/applicazione da parte del team incaricato e rappresenta un elemento fondamentale per la coordinazione dell’intera attività}
}

\newglossaryentry{front-end}
{
    name=\glslink{FRONT-END}{front-end},
    text=front-end,
    sort=front-end,
    description={E' la parte visibile all'utente di un programma e con cui egli può interagire, tipicamente un'interfaccia utente. E' responsabile dell'acquisizione dei dati di ingresso e della loro elaborazione con modalità conformi a specifiche predefinite e invarianti}
}

\newglossaryentry{crud}
{
    name=\glslink{crud}{CRUD},
    text=CRUD,
    sort=crud,
    description={Il termine si riferisce alle quattro operazioni basilari della gestione persistente dei dati}
}

\newglossaryentry{http}
{
    name=\glslink{http}{HTTP},
    text=HTTP,
    sort=http,
    description={E' un protocollo a livello applicativo usato come principale sistema per la trasmissione d'informazioni sul web ovvero in un'architettura tipica client-server}
}

\newglossaryentry{xml}
{
    name=\glslink{xml}{XML},
    text=XML,
    sort=xml,
    description={E' un metalinguaggio per la definizione di linguaggi di markup, ovvero un linguaggio basato su un meccanismo sintattico che consente di definire e controllare il significato degli elementi contenuti in un documento o in un testo}
}

\newglossaryentry{gui}
{
    name=\glslink{gui}{GUI},
    text=GUI,
    sort=gui,
    description={Denota l'interfaccia grafica. E' un tipo di interfaccia utente che consente l'interazione uomo-macchina in modo visuale utilizzando rappresentazioni grafiche (es. widget) piuttosto che utilizzando i comandi tipici di un'interfaccia a riga di comando}
}

\newglossaryentry{endpoint}
{
    name=\glslink{ENDPOINT}{endpoint},
    text=endpoint,
    sort=endpoint,
    description={E' un luogo digitale esposto tramite l'API dal quale l'API riceve le richieste e invia le risposte. Ogni endpoint è un URL (Uniform Resource Locator) che fornisce la posizione di una risorsa sul server dell'API}
}

\newglossaryentry{mock}
{
    name=\glslink{MOCK}{mock},
    text=mock,
    sort=mock,
    description={E' un oggetto simulato che riproduce il comportamento degli oggetti reali in modo controllato. Un programmatore crea un oggetto mock per testare il comportamento di altri oggetti, reali, ma legati ad un oggetto inaccessibile o non implementato. Allora quest'ultimo verrà sostituito da un mock}
}

\newglossaryentry{gps}
{
    name=\glslink{gps}{GPS},
    text=GPS,
    sort=gps,
    description={E' un sistema di posizionamento e navigazione satellitare militare statunitense}
}

\newglossaryentry{real-time system}
{
    name=\glslink{REAL-TIME SYSTEM}{real-time system},
    text=real-time system,
    sort=real-time system,
    description={E' un tipo di calcolatore in cui la correttezza del risultato delle sue computazioni dipende non solo dalla correttezza logica ma anche dalla correttezza temporale}
}

\newglossaryentry{orm}
{
    name=\glslink{orm}{ORM},
    text=ORM,
    sort=orm,
    description={E' una tecnica di programmazione mediante la quale gli oggetti ORM, mediante un'interfaccia orientata agli oggetti, forniscono tutti i servizi inerenti alla persistenza dei dati, astraendo nel contempo le caratteristiche implementative dello specifico database utilizzato}
}

\newglossaryentry{oop}
{
    name=\glslink{oop}{OOP},
    text=OOP,
    sort=oop,
    description={E' un paradigma di programmazione che permette di definire oggetti software in grado di interagire gli uni con gli altri attraverso lo scambio di messaggi}
}

\newglossaryentry{dto}
{
    name=\glslink{dto}{DTO},
    text=DTO,
    sort=dto,
    description={E' un design pattern usato per trasferire dati tra sottosistemi di un'applicazione software}
}

\newglossaryentry{test}
{
    name=\glslink{TEST}{test},
    text=test,
    sort=test,
    description={}
} % database di termini


%**************************************************************
% Impostazioni di graphicx
%**************************************************************
\graphicspath{{immagini/}} % cartella dove sono riposte le immagini


%**************************************************************
% Impostazioni di hyperref
%**************************************************************
\hypersetup{
    %hyperfootnotes=false,
    %pdfpagelabels,
    %draft,	% = elimina tutti i link (utile per stampe in bianco e nero)
    colorlinks=true,
    linktocpage=true,
    pdfstartpage=1,
    pdfstartview=,
    % decommenta la riga seguente per avere link in nero (per esempio per la stampa in bianco e nero)
    %colorlinks=false, linktocpage=false, pdfborder={0 0 0}, pdfstartpage=1, pdfstartview=FitV,
    breaklinks=true,
    pdfpagemode=UseNone,
    pageanchor=true,
    pdfpagemode=UseOutlines,
    plainpages=false,
    bookmarksnumbered,
    bookmarksopen=true,
    bookmarksopenlevel=1,
    hypertexnames=true,
    pdfhighlight=/O,
    %nesting=true,
    %frenchlinks,
    urlcolor=webbrown,
    linkcolor=RoyalBlue,
    citecolor=webgreen,
    %pagecolor=RoyalBlue,
    %urlcolor=Black, linkcolor=Black, citecolor=Black, %pagecolor=Black,
    pdftitle={\myTitle},
    pdfauthor={\textcopyright\ \myName, \myUni, \myFaculty},
    pdfsubject={},
    pdfkeywords={},
    pdfcreator={pdfLaTeX},
    pdfproducer={LaTeX}
}

%**************************************************************
% Impostazioni di itemize
%**************************************************************
\renewcommand{\labelitemi}{$\ast$}

%\renewcommand{\labelitemi}{$\bullet$}
%\renewcommand{\labelitemii}{$\cdot$}
%\renewcommand{\labelitemiii}{$\diamond$}
%\renewcommand{\labelitemiv}{$\ast$}


%**************************************************************
% Impostazioni di listings
%**************************************************************
\lstset{
    language=[LaTeX]Tex,%C++,
    keywordstyle=\color{RoyalBlue}, %\bfseries,
    basicstyle=\small\ttfamily,
    %identifierstyle=\color{NavyBlue},
    commentstyle=\color{Green}\ttfamily,
    stringstyle=\rmfamily,
    numbers=none, %left,%
    numberstyle=\scriptsize, %\tiny
    stepnumber=5,
    numbersep=8pt,
    showstringspaces=false,
    breaklines=true,
    frameround=ftff,
    frame=single
} 


%**************************************************************
% Impostazioni di xcolor
%**************************************************************
\definecolor{webgreen}{rgb}{0,.5,0}
\definecolor{webbrown}{rgb}{.6,0,0}


%**************************************************************
% Altro
%**************************************************************

\newcommand{\omissis}{[\dots\negthinspace]} % produce [...]

% eccezioni all'algoritmo di sillabazione
\hyphenation
{
    ma-cro-istru-zio-ne
    gi-ral-din
}

\newcommand{\sectionname}{sezione}
\addto\captionsitalian{\renewcommand{\figurename}{Figura}
                       \renewcommand{\tablename}{Tabella}}

\newcommand{\glsfirstoccur}{\ap{{[g]}}}

\newcommand{\intro}[1]{\emph{\textsf{#1}}}

%**************************************************************
% Environment per ``rischi''
%**************************************************************
\newcounter{riskcounter}                % define a counter
\setcounter{riskcounter}{0}             % set the counter to some initial value

%%%% Parameters
% #1: Title
\newenvironment{risk}[1]{
    \refstepcounter{riskcounter}        % increment counter
    \par \noindent                      % start new paragraph
    \textbf{\arabic{riskcounter}. #1}   % display the title before the 
                                        % content of the environment is displayed 
}{
    \par\medskip
}

\newcommand{\riskname}{Rischio}

\newcommand{\riskdescription}[1]{\textbf{\\Descrizione:} #1.}

\newcommand{\risksolution}[1]{\textbf{\\Soluzione:} #1.}

%**************************************************************
% Environment per ``use case''
%**************************************************************
\newcounter{usecasecounter}             % define a counter
\setcounter{usecasecounter}{0}          % set the counter to some initial value

%%%% Parameters
% #1: ID
% #2: Nome
\newenvironment{usecase}[2]{
    \renewcommand{\theusecasecounter}{\usecasename #1}  % this is where the display of 
                                                        % the counter is overwritten/modified
    \refstepcounter{usecasecounter}             % increment counter
    \vspace{10pt}
    \par \noindent                              % start new paragraph
    {\large \textbf{\usecasename #1: #2}}       % display the title before the 
                                                % content of the environment is displayed 
    \medskip
}{
    \medskip
}

\newcommand{\usecasename}{UC}

\newcommand{\usecaseactors}[1]{\textbf{\\Attori Principali:} #1. \vspace{4pt}}
\newcommand{\usecasepre}[1]{\textbf{\\Precondizioni:} #1. \vspace{4pt}}
\newcommand{\usecasedesc}[1]{\textbf{\\Descrizione:} #1. \vspace{4pt}}
\newcommand{\usecasepost}[1]{\textbf{\\Postcondizioni:} #1. \vspace{4pt}}
\newcommand{\usecasealt}[1]{\textbf{\\Scenario Alternativo:} #1. \vspace{4pt}}

%**************************************************************
% Environment per ``namespace description''
%**************************************************************

\newenvironment{namespacedesc}{
    \vspace{10pt}
    \par \noindent                              % start new paragraph
    \begin{description} 
}{
    \end{description}
    \medskip
}

\newcommand{\classdesc}[2]{\item[\textbf{#1:}] #2}